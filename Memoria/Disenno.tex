
\chapter{DISEÑO}

\section{Estructura}

Seguidamente puede ver el código principal para la resolucón del laberinto:

\inputminted{arduino}{code/main.ino}

\vspace{-15pt}
\section{Plan de pruebas}
Se ha implementado una pequeña función que nos permite, a partir mediante Bluetooth conocer el estado de los sensores y poder actuar en consecuencia si las mediciones son anormales.

\begin{minted}{cpp}
void testSensors() {
  if (robot.ReadCNY('L'))
    Serial1.println("CNY L Black");
  else
    Serial1.println("CNY L White");
  if (robot.ReadCNY('R'))
    Serial1.println("CNY R Black");
  else
    Serial1.println("CNY R White");
  if (robot.ReadCNY('B'))
    Serial1.println("CNY B Black");
  else
    Serial1.println("CNY B White");
  Serial1.print("Ultrasonic: "); Serial1.println(robot.ReadUltrasonic());
  Serial1.print("Sharp L: "); Serial1.println(robot.ReadSharp('L'));
  Serial1.print("Sharp R: "); Serial1.println(robot.ReadSharp('R'));
}
\end{minted}
